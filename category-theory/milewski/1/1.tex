\documentclass[titlepage, 12pt]{article}
\usepackage[parfill]{parskip}
\usepackage{amsmath}

\begin{document}

\title{Category Theory Unit 1}

\author{Bharathi Ramana Joshi}

\date{\today}

\maketitle

\newpage

\section{Book}

\begin{enumerate}
	\item A category consists of \textit{objects} and \textit{arrows} that go
		between them. The essence of a category is \textit{composition}.
	\item Two properties of any category are
		\begin{enumerate}
			\item Composition is associative ($h(g(f)) == (h(g))f$)
			\item Existence of unit composition ($f\circ\textbf{id}_A = f$ and
				$\textbf{id}_B\circ f = f$)
		\end{enumerate}
\end{enumerate}

\section{Lecture 1}

\begin{enumerate}
	\item OOPS problem: data race - mutation vs data sharing. No
		concurrency => no composition.
	\item Assembly -> Procedural -> OOPS -> Functional
	\item Category theory - language to express highly abstract ideas
	\item Complex situation? Decompose into smaller solutions
	\item Simpler chunks should be simpler to solve
\end{enumerate}

\section{Lecture 2}

\subsection{Abstraction}

\begin{enumerate}
	\item Abstraction - get rid of unnecessary details
	\item Equal (exactly) or almost equal (for all intents and purposes) -
		homotopy type theory?
\end{enumerate}

\subsubsection{Category}
Categories deal with
\begin{enumerate}

	\item Composition

	\item Identity

\end{enumerate}

A category is defined as
\begin{enumerate}
	\item A "bunch" of objects (dots)
	\item Morphisms (arrows between dots)
\end{enumerate}

An object is a primitive - no structure - atomic

Arrows should be such that
\begin{gather*}
	a\rightarrow b \textrm{ and } b\rightarrow c \Rightarrow a\rightarrow
	c\textrm{ (transitivity) }\\
	\forall a, a\rightarrow a\textrm{ (identity) }\\
	h\odot(g\odot f) = (h\odot g)\odot f\textrm{ (associativity) }
\end{gather*}

\verb#Set# - consider a function form set $A$ to set $B$. Now abstract out the
elements of $A$ and $B$. Just look at the mappings. We have a category - the
objects are $A$ and $B$ and the morphisms are the mappings.

Question: If a morphism is defined by which two objects they connect and contain
no inner structure, how could there be more than a single arrow between two
objects? 
Answer:  This is an interesting question that a lot of people are asking. We are
so used to identifying things by their internal structure. In category theory
things don't have internal structure. Their identity is defined by the way they
interact with other things. In this case, a particular morphism is not only
defined by its endpoints, but also by how it composes with all the morphisms
around it. Tell me how you compose and I'll tell you who you are. 

\end{document}

