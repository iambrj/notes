\documentclass[titlepage, 12pt]{article}
\usepackage{amsmath}
\usepackage[parfill]{parskip}
\usepackage{proof}
\usepackage{xcolor}
\usepackage{amsfonts}
\usepackage{setspace}
\usepackage{hyperref}
\usepackage{tcolorbox}
\usepackage{amssymb}
\tcbuselibrary{theorems}

\hypersetup{
    colorlinks=true,
    linkcolor=blue,
    filecolor=magenta,
    urlcolor=blue,
}


\newcommand{\xRightarrow}[2][]{\ext@arrow 0359\Rightarrowfill@{#1}{#2}}
\makeatletter
\newcommand\xleftrightarrow[2][]{%
\ext@arrow 9999{\longleftrightarrowfill@}{#1}{#2}}
\newcommand\longleftrightarrowfill@{%
\arrowfill@\leftarrow\relbar\rightarrow}
\makeatother

\newtcbtheorem[]{definition}{Definition}%
{colback=magenta!5,colframe=magenta!100!black,fonttitle=\bfseries}{th}

\newtcbtheorem[]{proposition}{Proposition}%
{colback=cyan!5,colframe=cyan!100!black,fonttitle=\bfseries}{th}

\newtcbtheorem[]{theorem}{Theorem}%
{colback=orange!5,colframe=orange!100!black,fonttitle=\bfseries}{th}

\begin{document}

\begin{titlepage}

	\raggedleft

	\vspace*{\baselineskip}

	{Bharathi Ramana Joshi\\\url{https://github.com/iambrj/notes}}

	\vspace*{0.167\textheight}

	\textbf{\LARGE Notes on}\\[\baselineskip]

	\textbf{\textcolor{teal}{\huge Abstract Reduction Systems}}\\[\baselineskip]

    {\Large \textit{Chapter 1 from Term Rewriting and All That}}

	\vfill

	\vspace*{3\baselineskip}

\end{titlepage}

\section{Equivalence and reduction}

\begin{itemize}

  \item\textbf{Abstract reduction system}: pair $(A,\rightarrow)$ such that
    $\rightarrow\subseteq A\times A$
\end{itemize}
Two ways 
Some definitons

\begin{center}
  \begin{tabular}{c c}

    $\xrightarrow{0} := \{(x, x)\mid x\in A\}$ & \textbf{identity}\\

    $\xrightarrow{i+1} := \xrightarrow{i}\circ\rightarrow$ &
    ($i+1$)-\textbf{fold composition}, $i\ge 0$\\

    $\xrightarrow{+} := \cup_{i\ge 0}\xrightarrow{i}$ & \textbf{transitive
    closure}]\\

      $\xrightarrow{*} := \xrightarrow{+}\cup\xrightarrow{0}$ &
      \textbf{reflexive transitive closure}\\

      $\xrightarrow{=} := \rightarrow\cup\xrightarrow{0}$ & \textbf{reflexive
      closure}\\

        $\xrightarrow{-1} := \{(y, x)\mid x\rightarrow y\}$ & \textbf{inverse}\\

        $\leftarrow := \xrightarrow{-1}$ & \textbf{inverse}\\

        $\leftrightarrow := \rightarrow\cup\leftarrow$ & \textbf{symmetric
        closure}\\

          $\xleftrightarrow{+} := (\leftrightarrow)^+$ & \textbf{transitive symmetric
          closure}\\

            $\xleftrightarrow{*} := (\leftrightarrow)^*$ & \textbf{reflexive transitive
            symmetric closure}

  \end{tabular}
\end{center}

\begin{itemize}

  \item $x$ is \textbf{reducible} iff there is a $y$ such that $x\rightarrow y$

  \item $x$ is in \textbf{normal form} iff it is not reducible

  \item $y$ is \textbf{a normal form} of $x$ iff $x\xrightarrow{*}y$ and $y$ is
    in normal form. If $x$ has a uniquely determined normal form, the latter is
    denoted by $x\downarrow$

  \item $y$ is a \textbf{direct successor} of $x$ iff $x\rightarrow y$

  \item $y$ is a \textbf{successor} of $x$ iff $x\xrightarrow{+}y$

  \item $x$ and $y$ are \textbf{joinable} (written $x\downarrow y$) iff there is a $z$ such that
    $x\xrightarrow{*}z\xleftarrow{*}y$

\item $x\uparrow y$ iff $\exists z$ such that $z\xrightarrow{*} x$ and
    $z\xrightarrow{*} y$

\end{itemize}

A reduction $\rightarrow$ is called

\begin{itemize}

  \item\textbf{Church-Rosser} iff $x\xleftrightarrow{*}y\implies x\downarrow y$

  \item\textbf{confluent} iff $y_1\xleftarrow{*} x\xrightarrow{*}
      y_2\implies y_1\downarrow y_2$

  \item\textbf{semi-confluent} iff $y_1\leftarrow x\xrightarrow{*}y_2\implies
    y_1\downarrow y_2$

  \item\textbf{terminating} iff there is no infinite descending chain
    $a_0\rightarrow a_1\rightarrow\dots$

  \item\textbf{normalizing} iff every element has a normal form

  \item\textbf{convergent} iff it is both confluent and terminating

\end{itemize}
\textbf{Theorem.} Church-Rosser, confluence and semi-confluence are equivalent\\
\textbf{Proof.} Use $1 \implies 2 \implies 3 \implies 1$

Useful results
\begin{enumerate}
    \item If $\rightarrow$ is confluent and $x\xleftrightarrow{*}$ y then
        \begin{enumerate}
            \item $x\xrightarrow{*} y$ if $y$ is in normal form
            \item $x = y$ if both $x$ and $y$ are in normal form
        \end{enumerate}
    \item If $\rightarrow$ is confluent then every element has at most normal form
    \item If $\rightarrow$ is normalizing and confluent, every element has a
        unique normal form
\end{enumerate}

\section{Well-founded induction}
\infer{\forall x\in A. P(x)}{\forall x\in A.(\forall y\in A.x\xrightarrow{+}y\implies P(y))\implies P(x)}
No base case you say? The premise of WFI subsumes the base case i.e. all terms
without a successor should be proved individually.

\textbf{Theorem.} $\rightarrow$ terminates iff WFI holds.\\
\textbf{Proof.} Forward by contradiction and backward by application
Some more definitons - a relation $\rightarrow$ is called
\begin{itemize}
    \item\textbf{finitely branching} if each element has only finitely many
        direct successors
    \item\textbf{globally finite} if each element has only finitely many
        successors
    \item\textbf{acyclic} if there is no element such that $a\xrightarrow{*}a$
\end{itemize}
\textbf{K\"onig's Lemma}: A finitely branching tree is infinite iff it contains
an infinite path
\end{document}

