\documentclass[titlepage, 12pt]{article}
\usepackage[parfill]{parskip}
\usepackage{amsmath}
\usepackage{amsfonts}
\usepackage{mdframed}
\usepackage{xcolor}

\newenvironment{definition}[1][Definition]
    {\leavevmode \\ \begin{mdframed}[backgroundcolor=gray!20] \textbf{#1} \\}
    {  \end{mdframed}\leavevmode \\}

\begin{document}

\title{Notes on subgroups}

\author{Bharathi Ramana Joshi}

\date{Compiled: \today}

\maketitle

\newpage

\section{Definiton and examples}
\begin{definition}
    If $G$ is a group and $H$ is a subset of $G$ such that
    \begin{enumerate}
        \item $H$ is nonempty
        \item $H$ is closed under $G$'s group operator (i.e. $x, y\in H\implies
            xy\in H$)
        \item $H$ is closed under inverses (i.e. $x\in H \implies x^{-1} \in H$)
    \end{enumerate}
\end{definition}

\textbf{Subgroup criterion}: A subset $H$ of a group $G$ is a subgroup iff
\begin{enumerate}
    \item $H\neq\phi$
    \item $\forall x, y\in H, xy^{-1}\in H$
\end{enumerate}
Furthermore, if $H$ is finite it sufficies to check that $H$ is nonempty and
closed under multiplication.

\begin{definition}
    $C_G(A) = \{g\in G| gag^{-1} = a,\forall a\in A\}$ is a subset of $G$ and
    is called the \textbf{centralizer} of $A$ in $G$. Since $gag^{-1} = a$ iff
    $ga = ag$, $C_G(A)$ is the set of elements of $G$ which commute with every
    element of $A$.
\end{definition}
Intuitively, the centralizer of $A$ measures how inside the center $Z(G)$ $A$
is.

\begin{definition}
    The sbuset $Z = \{g\in G| gx = xg\forall x\in G\}$ of $G$ is called the
    \textbf{center} of $G$.
\end{definition}
The center of a set is just the centralizer with $A = G$

\begin{definition}
    If $gAg^{-1} = \{gag^{-1}| a\in A\}$, the \textbf{normalizer}  of $A$ in
    $G$ is defined as $N_G(A) = \{g\in G| gAg^{-1} = A \}$
\end{definition}
Intuitively, the normalizer of a set $A$ measures how normal the set $A$ is.


On a side note, observe that if $g\in C_G(A)$ then $gag^{-1} = a\in A$ for all
$a\in A$ so $C_G(A)\le N_G(A)$

\begin{definition}
    If $G$ is a group an $S$ is a set on which $G$ acts, then for any fixed
    $s\in S$ the \textbf{stabilizer} of $s$ is defined as $\{g\in G| g.s=s\}$
\end{definition}

\begin{definition}
    The kernel of the action of $G$ on $S$ is defined as $\{g\in G| g.s = s,
    \forall s\in S\}$
\end{definition}

\begin{definition}
    A group $H$ is \textbf{cyclic} if $H$ can be generated by a single element,
    i.e. $\exists x\in H$ such that $H = \{x^n| n\in \mathbb{Z}\}$. This is denoted
    using $H = \langle x\rangle$ and $x$ is said to generate $H$.
\end{definition}
\begin{enumerate}
    \item Cyclic groups are abelian
    \item If $H = \langle x\rangle$, then $| H| = | x|$
    \item For any element $x\in G$ of an arbitrary group $G$, and $m, n\in
        \mathbb{Z}$,
        if $x^n = 1$ and $x^m = 1$, then $x^{GCD(m, n)} = 1$. In particular, if
        $x^m = 1$ for some $m\in \mathbb{Z}$, then $|x|$ divides $m$.
    \item Any two cyclic groups of same order are isomorphic
    \item For any element $x\in G$ of an arbitrary group $G$, and $a\in
        \mathbb{Z} - \{0\}$
        \begin{enumerate}
            \item If $|x| = \infty$, then $|x^a| = \infty$.
            \item If $|x| = n < \infty$, then $|x^a| = \frac{n}{(n, a)}$.
            \item In particular, if $|x| = n < \infty$ and $a$ is a positive integer
                dividing $n$, then $|x^a| = \frac{n}{a}$.
        \end{enumerate}
    \item If $H = |x|$ then
        \begin{enumerate}
            \item If $|x| = \infty$ then $H = \langle x^a\rangle$ iff $a =
                \pm 1$
            \item If $|x| = n < \infty$ then $H = \langle x^a\rangle$ iff $(a, n) =
                1$. In particular the number of generators of $H$ is $\phi(n)$
        \end{enumerate}
    \item If $H = \langle x\rangle$ is a cyclic group then
        \begin{enumerate}
            \item Every subgroup of $H$ is cyclic, more precisely if $K\le H$,
                then either $K = \{1\}$ or $K = \{x^d\}$, where $d$ is the
                smallest positive integer such that $x^d\in K$.
            \item If $|H| = \infty$ then for all $a, b\in\mathbb{Z}$ such that
                $a\neq b$, $\langle x^a\rangle\neq\langle x^b\rangle$ and for
                every $m\in\mathbb{Z}$, $\langle x^m\rangle = \langle
                |x^m|\rangle$, i.e.\ nontrivial subgroups of $H$ are in a
                bijective correspondence with the natural numbers.
            \item If $|H| = n < \infty$ then for all $a\in\mathbb{Z}$ such that
                $a|n$, $\exists$ unique subgroup $H$ of order $a$ (which is the
                cyclic group $\langle x^{\frac{n}{a}}\rangle$), and for every
                $m\in\mathbb{Z}$, $\langle x^m\rangle = \langle x^{(n,
                m)}\rangle$, i.e.\ subgroups of $H$ are in a bijective
                correspondence with the positive divisors of $n$.
        \end{enumerate}
    \item If $\mathcal{A}$ is any nonempty collection of subgroups of $G$, then
        the intersection of all members of $\mathcal{A}$ is also a subgroup of
        $G$
    \item $\bar{A} = \langle A\rangle$
\end{enumerate}

\end{document}
