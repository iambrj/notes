\documentclass[titlepage, 12pt]{article}
\usepackage[parfill]{parskip}
\usepackage{wasysym}
\usepackage{pifont}

\begin{document}

\title{Abstract Algebra: Chapter 1}

\author{Bharathi Ramana Joshi}

\date{\today}

\maketitle

\tableofcontents

\newpage


\section{Basic Axioms}

\begin{enumerate}

	\item A \textit{binary operation} \ding{72} on a set $G$ is a function
		\ding{72}:$G\times G\rightarrow G$.

	\item A binary operation \ding{72} on a set $G$ is \textit{associative} if
		$\forall a, b, c\in G$ we have $a$\ding{72}$(b$\ding{72}$c)$ =
		$(a$\ding{72}$ b)$\ding{72}$ c$

	\item If \ding{72} is a binary operator on set $G$, we say elements $a$
		and $b$ of $G$ \textit{commute} if $a$\ding{72}$b$  = $b$\ding{72}$a$.
		We say \ding{72} is commutative if for all $a, b\in G$, $a$\ding{72}$b =
		b$\ding{72}$a$.

\end{enumerate}

Definitions

\begin{enumerate}

	\item A group is an ordered pair $(G,$\ding{72}$)$, where $G$ is a set and
		\ding{72} is a binary operator such that

		\begin{enumerate}

			\item \ding{72} is closed under $G$

			\item $a$\ding{72}$(b$\ding{72}$c)$ = $(a$\ding{72}$b)$\ding{72}$c$
				- \ding{72} is associative

			\item $\exists e\in G$ such that $\forall a\in G$ we have
				$a$\ding{72}$e$ = $e$\ding{72}$a$ = a - identity existence

			\item $\forall a\in G$, $\exists a^{-1}\in G$ such that
				$a$\ding{72}$a^{-1}$ = $a^{-1}$\ding{72}$a = e$ - inverse
				existence

		\end{enumerate}

	\item A group $(G,$ \ding{72}) is called an \textit{Abelian/Commutative} if
		it also satisfies the commutative property - $a$\ding{72}$b$ =
		$b$\ding{72}$a$, $\forall a, b\in G$

	\item For a group $G$ and $x\in G$ the \textit{order} of $x$ is the smallest
		positive integer $n$ such that $x^n = 1$, denoted by $|x|$. $n$ is
		called the order of $x$. If no positive $n$ exists, $x$ is said to be of
		order infinity.

	\item Let $G = {g_1, g_2,\ldots}$ be a finite group with $g_1 = 1$. The
		\textit{multiplication table}/\textit{group table} of $G$ is the
		$n\times n$ matrix whose $i, j$ entry is the group element $g_ig_j$.

\end{enumerate}

\end{document}

