\documentclass[titlepage, 12pt]{article}
\usepackage[parfill]{parskip}
\usepackage{amsmath}
\usepackage{xcolor}
\usepackage{amsfonts}
\usepackage{setspace}
\usepackage{hyperref}
\usepackage{tcolorbox}
\tcbuselibrary{theorems}

\hypersetup{
    colorlinks=true,
    linkcolor=blue,
    filecolor=magenta,
    urlcolor=blue,
}

\newtcbtheorem[]{definition}{Definition}%
{colback=magenta!5,colframe=magenta!100!black,fonttitle=\bfseries}{th}

\newtcbtheorem[]{proposition}{Proposition}%
{colback=cyan!5,colframe=cyan!100!black,fonttitle=\bfseries}{th}

\newtcbtheorem[]{theorem}{Theorem}%
{colback=orange!5,colframe=orange!100!black,fonttitle=\bfseries}{th}

\begin{document}

\begin{titlepage}

	\raggedleft

	\vspace*{\baselineskip}

	{Bharathi Ramana Joshi\\\url{https://github.com/iambrj/notes}}

	\vspace*{0.167\textheight}

	\textbf{\LARGE Notes on}\\[\baselineskip]

	\textbf{\textcolor{teal}{\huge Introduction to Rings}}\\[\baselineskip]

    {\Large \textit{Chapter 7 from Dummit \& Foote, $3^{rd}$ Ed.}}

	\vfill

	\vspace*{3\baselineskip}

\end{titlepage}

\newpage

\begin{definition}{Ring}{}
    A \textbf{ring} $R$ is a set with two binary operations $+$ and $\times$
    such that
    \begin{enumerate}
        \item $(R, +)$ is an abelian group
        \item $\times$ is associative : $(a\times b)\times c = a\times (b\times
            c)$
        \item The following distributive laws hold
            \begin{enumerate}
                \item $(a+b)\times c = (a\times c) + (b\times c)$
                \item $a\times (b+c) = (a\times b) + (a\times c)$
            \end{enumerate}
    \end{enumerate}
\end{definition}

A ring is commutative if multiplication is commutative

A ring is said to have an identity if $\exists 1\in R$ such that
\begin{gather*}
    1\times a = a\times 1 = a, \forall a\in R
\end{gather*}

\begin{definition}{Division ring}{}
    A ring $R$ with an identity element 1 ($\neq 0$) is called a
    \textbf{division ring/skew field} if inverses exist for all nonzero elements
    such that $ab=ba=1$
\end{definition}

A commutative division ring is called a \textbf{field}

\begin{proposition}{}{}
    If $R$ is a ring then
    \begin{enumerate}
        \item $0a=a0=0, \forall a\in R$
        \item $(-a)b=a(-b)=-(ab), \forall a,b\in R$
        \item $(-a)(-b) = ab,\forall a,b\in R$
        \item If $R$ has identity 1, then it is unique and $-a = (-1)a$
    \end{enumerate}
\end{proposition}

\begin{definition}{Zero divisor, unit}{}
    \begin{enumerate}
        \item A nonzero element $a\in R$ is called a \textbf{zero divisor} if
        $\exists b\neq 0\in R$ such that $ab=0$ or $ba=0$
        \item If $R$ has an identity $1\neq 0$, then $u\in R$ is called
        \textbf{unit} in $R$ if $\exists v\in R$ such that $uv=vu=1$. Set of
        units is denoted by $R^\times$
    \end{enumerate}
\end{definition}
 Note:
\begin{enumerate}
    \item The units in a ring $R$ form a group under multiplication
    \item A zero divisor can never be a unit
\end{enumerate}

\begin{definition}{Integral domain}{}
    A commutative ring with identity $1\neq 0$ is called an \textbf{integral
    domain} if it has no zero divisors
\end{definition}

\begin{proposition}{}{}
    If $a,b,c\in R$ and $a$ not a zero divisor, then $ab=ac\implies a = 0 \lor b =
    c$, equivalent $ab=ac\implies a = 0\lor b = c$
\end{proposition}

Note that any finite integral domain is a field since for any nonzero $a\in R$.
$aR$ is a bijective map

\begin{definition}{Subring}{}
    A \textbf{subring} of a ring $R$ is a subgroup of $R$ closed under
    multiplication
\end{definition}

\begin{definition}{Ring homomorphism, kernel}{}
    If $R$ and $S$ are rings then
    \begin{enumerate}
        \item A ring homomorphism is a map $\phi:R\rightarrow S$ such that
        \begin{enumerate}
            \item $\phi(a+b) = \phi(a) + \phi(b)$
            \item $\phi(ab) = \phi(a)\phi(b)$
        \end{enumerate}
        \item The kernel of $\phi$, denoted $ker\phi$ is the set of $r\in R$
            such that $\phi(r) = 0_S$
        \item A bijective ring homomorphism is called an isomorphism
    \end{enumerate}
\end{definition}

\begin{proposition}{}{}
    If $R$ and $S$ are two rings and $\phi:R\rightarrow S$ is a homomorphism,
    then
    \begin{enumerate}
        \item The image of $\phi$, $\phi(R)$ is a subring of $S$
        \item The kernel of $\phi$, $ker\phi$ is a subring of $R$
    \end{enumerate}
\end{proposition}

\begin{definition}
    If $R$ is a ring, $I\subseteq R$ and $r\in R$
    \begin{enumerate}
        \item $rI = \{ra|a\in I\}$ and $Ir = \{ar|a\in I\}$
        \item $I$ is a left ideal if
            \begin{enumerate}
                \item $I$ is a subring of $R$
                \item $I$ is a closed under left multiplication by elements of $R$
            \end{enumerate}
        \item $I$ is a right ideal if
            \begin{enumerate}
                \item $I$ is a subring of $R$
                \item $I$ is a closed under right multiplication by elements of $R$
            \end{enumerate}
        \item If $I$ is both a left and right ideal, then it is called a
            two-sided ideal
    \end{enumerate}
\end{definition}

\begin{proposition}{}{}
    If $I$ is an ideal of the ring $R$ then the additive quotient group $R/I$ is
    a ring, called the \textbf{quotient right} under the following binary
    operations
    \begin{gather*}
        (r + I) + (s + I) = (r + s) + I\\
        (r + I) \times (s + I) = (rs) + I\\
    \end{gather*}
\end{proposition}

\begin{theorem}{First Isomorphism Theorem for Rings}{}
    If $\phi:R\rightarrow S$ is a homomorphism of rings then
    \begin{enumerate}
        \item $ker\phi$ is an ideal of $R$
        \item $\phi(R)$ is isomorphic to a subring of $S$
        \item $R/ker\phi$ is isomorphic to $\phi(R)$
    \end{enumerate}
\end{theorem}

\begin{definition}{}{}
    If $I$ is any ideal of the ring $R$, then the map
    \begin{gather*}
        \phi:R\rightarrow S, r\rightarrow r+I
    \end{gather*}
    is a surjective ring homomorphism, called the natural projection of $R$ onto
    $R/I$
\end{definition}

\begin{theorem}{Second Isomorphism Theorem for Rings}{}
    If $S$ is a subring and $I$ is an ideal of a ring $R$ then $\{s + i|s\in S,
    i\in I\}$ is a subring, $S\cap I$ is an ideal of $S$ and $(S+I)/I\cong
    S/(S\cap I)$
\end{theorem}

\begin{theorem}{Third Isomorphism Theorem for Rings}{}
    If $I$ and $J$ are ideals of the ring $R$ such that $I\subseteq J$ then
    $J/I$ is an ideal of $R/I$ and $(R/I)/(J/I)\cong R/J$
\end{theorem}

\begin{theorem}{Fourth Isomorphism Theorem for Rings}{}
    If $I$ is an ideal of a ring $R$ then the correspondence $A\leftrightarrow
    R/I$ is an inclusion preserving bijection between set of subrings $A$ of $R$
    containing $I$ and set of subrings $R/I$
\end{theorem}

\begin{definition}{}{}
    Let $A$ be any subset of the ring $R$
    \begin{enumerate}
        \item The smallest ideal of $R$ containing $A$ is called the ideal
            generated by $A$, denoted by $(A)$
        \item $RA$ denotes the set of all finite sums of elements of the form
            $ra$ (where $r\in R$) i.e.,
            \begin{gather*}
                RA = \{a_1r_1+\dots+a_nr_n | a_i\in A, r_i\in R,
                n\in\mathbb{Z}^+\}
            \end{gather*}
        \item An ideal generated by a single element is called a
            \textbf{principal ideal}
        \item An ideal generated by a finite set is called a \textbf{finitely
            generated ideal}
    \end{enumerate}
\end{definition}

\begin{proposition}{}{}
    If $I$ is an ideal of the ring $R$
    \begin{enumerate}
        \item $I = R\iff I$ has a unit
        \item If $R$ is commutative, then $R$ is a field iff its only ideals are
            $0$ and $R$
    \end{enumerate}
\end{proposition}

\textbf{Corollary}: If $R$ is a field then any nonzero ring homomorphism from
$R$ into another ring is an injection

\begin{definition}{}{}
    An ideal $M$ in an arbitrary ring $S$ is called \textbf{maximal ideal} if
    $M\neq S$ and the only ideals containing $M$ are $M$ and $S$
\end{definition}

\begin{proposition}{}{}
    In a ring with identity every proper ideal is contained in a maximal ideal
\end{proposition}

\begin{proposition}{}{}
    If $R$ is commutative, then an ideal $M$ is maximal iff the quotient ring
    $R/M$ is a field
\end{proposition}

\begin{definition}{}{}
    If $R$ is commutative, an ideal $P$ is called a \textbf{prime ideal} if
    $P\neq R$ and whenever the product $ab$ of two elements $a, b\in R$ is in
    $P$, then at least one of $a$ and $b$ is in $P$
\end{definition}

\begin{proposition}{}{}
    If $R$ is commutative, then an ideal $P$ is prime in $R$ iff the quotient
    ring $R/P$ is an integral domain
\end{proposition}

\textbf{Corollary}: If $R$ is commutative, every maximal ideal of $R$ is prime

\end{document}

