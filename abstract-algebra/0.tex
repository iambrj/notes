\documentclass[titlepage, 12pt]{article}
\usepackage[parfill]{parskip}
\usepackage{amsfonts}

\begin{document}

\title{Abstract Algebra: Chapter 0}

\author{Bharathi Ramana Joshi}

\date{\today}

\maketitle

\tableofcontents

\newpage

\section{Basics}
A function $f:A\rightarrow B$
\begin{enumerate}

	\item is injective if $\forall a_1\neq a_2$,
		$f(a_1)\neq f(a_2)$.

	\item is surjective if range = co-domain
		($\forall b\in B, \exists a\in A$ such that $f(a) = b$).

	\item is bijective if it is both injective and
		surjective.

	\item has a \textit{left inverse} if $\exists g:B\rightarrow A$ such that
		$g\circ f:A\rightarrow A$ is the identity map on $A$.

	\item has a \textit{right inverse} if $\exists h:B\rightarrow A$ such that
		$f\circ h:B\rightarrow B$ is the identity map on $B$.

\end{enumerate}

A binary relation

\begin{enumerate}

	\item is defined as a subset $R$ of $A\times A$ and written as $a\sim b$ if
		$(a, b)\in R$.

	\item is \textit{reflexive} if $\forall a\in A$, $a\sim a$.

	\item is \textit{symmetric} if $\forall a\sim b$, $b\sim a$.

	\item is \textit{transitive} if $\forall a\sim b$ and $b\sim c$,
		$a\sim c$.
 
	\item is a \textit{equivalence relation} if it is reflexive, symmetric and
		transitive.

\end{enumerate}

If $\sim$ is an equivalence relation on $A$, then the \textit{equivalence class}
of $a\in A$ is defined as $\{x\in A\mid x\sim a\}$

If $C$ is an equivalence class, any element of $C$ is called a
\textit{representative} element of the class $C$.

A \textit{partition} of $A$ is any collection nonempty subsets $\{A_i\mid i\in
I\}$ (for some indexing set $I$) such that

	\begin{enumerate}

		\item $A = \cap_{i\in I} A_i$

		\item $A_i\cap A_j = \phi$, $\forall i, j\in I$ such that $i\neq j$

	\end{enumerate}

	Notions of equivalence class and partition are the same (proof idea: ).

	\section{Properties of Integers}

	\begin{enumerate}

		\item\textit{Well Ordering}: If $A$ is any nonempty subset of
			$\mathbb{Z}$ then $\exists m\in A$ such that $m\le a$, $\forall a\in
			A$

		\item\textit{Euler Totient Function}: for $n\in\mathbb{Z}$, $\phi(n)$ is
			the number of positive integers $a\le n$ that are relative prime to
			$n$

	\end{enumerate}

\end{document}

